
\documentclass[aps,pra,onecolumn,notitlepage,superscriptaddress]{revtex4-1}

%\input{myQcircuit}
\usepackage{graphicx,color}% Include figure files
\usepackage{dcolumn}% Align table columns on decimal point
\usepackage{bm}% bold math
\usepackage{amsmath,amssymb,mathrsfs}
\usepackage{url}
\usepackage{hyperref}

% added by me
\usepackage{framed}
\usepackage{algorithm}
\usepackage{dsfont}
\usepackage{mathtools}
\DeclarePairedDelimiter{\ceil}{\lceil}{\rceil}
\DeclarePairedDelimiter{\floor}{\lfloor}{\rfloor}


\newcommand{\N}{\mathbb{N}}
\newcommand{\Z}{\mathbb{Z}}
\newcommand{\R}{\mathbb{R}}
\newcommand{\C}{\mathbb{C}}


%  Sets
\newcommand{\set}[1]{\mathsf{#1}}
\newcommand{\grp}[1]{\mathsf{#1}}
\newcommand{\reg}[1]{\mathsf{#1}}
\newcommand{\spc}[1]{\mathcal{#1}}

% Integrals

\def\d{{\rm d}}

% Linear structures
\newcommand{\Span}{{\mathsf{Span}}}
\newcommand{\Lin}{\mathsf{Lin}}
\newcommand{\Pos}{\mathsf{Pos}}
\newcommand{\CP}{\mathsf{CP}}
\newcommand{\Herm}{\mathsf{Herm}}
\newcommand{\D}{\mathsf{D}}
\newcommand{\Proj}{\mathsf{Proj}}
\newcommand{\U}{\mathsf{U}}
\newcommand{\Diag}{\mathsf{Diag}}
% added by me
\newcommand{\T}{\mathsf{T}}
\newcommand{\F}{\mathsf{F}}
\newcommand{\ca}{\mathsf{C}}

% added by me
\newcommand{\rank}{\mathsf{rank}}
\newcommand{\im}{\mathsf{im}}
\newcommand{\myker}{\mathsf{ker}}
% \newcommand{\Pr}{\mathsf{Pr}}

\def\>{\rangle}
\def\<{\langle}
\def\kk{\>\!\>}
\def\bb{\<\!\<}
\newcommand{\st}[1]{\mathbf{#1}}
\newcommand{\bs}[1]{\boldsymbol{#1}}

% Linear maps
\newcommand{\map}[1]{\mathcal{#1}}
\newcommand{\Tr}{\operatorname{Tr}}
\newcommand{\diag}{\mathsf{diag}}


%  Operational notions
\newcommand{\op}[1]{\operatorname{#1}}

\newcommand{\St}{{\mathsf{St}}}
\newcommand{\Eff}{{\mathsf{Eff}}}
\newcommand{\Pur}{{\mathsf{Pur}}}
\newcommand{\Transf}{{\mathsf{Transf}}}
\newcommand{\Chan}{{\mathsf{Chan}}}


%   By Mo
\newcommand{\arccot}{\mathrm{arccot}\,}

%  Miscellanea
\newcommand\myuparrow{\mathord{\uparrow}}
\newcommand\mydownarrow{\mathord{\downarrow}}
\newcommand\h{{\scriptstyle \frac 12}}
% added by me
\newcommand\I{\mathds{1}}

\newcommand{\vertiii}[1]{{\left\vert\kern-0.25ex\left\vert\kern-0.25ex\left\vert #1 
    \right\vert\kern-0.25ex\right\vert\kern-0.25ex\right\vert}}

% Environments
\newtheorem{theo}{Theorem}
\newtheorem{ax}{Axiom}
\newtheorem{lemma}{Lemma}
\newtheorem{prop}{Proposition}
\newtheorem{cor}{Corollary}
\newtheorem{defi}{Definition}


\newtheorem{rem}{Remark}
\newtheorem{ex}{Exercise}

\newtheorem{proper}{Property}

\def\Proof{{\bf Proof.~}}
\def\qed{$\blacksquare$ \newline}

\begin{document}
    \preprint{APS/123-QED}
    \title{Quantum state discrimination}
    \author{}
    \maketitle
    % \tableofcontents
    % \newpage

    \section{Classical information over quantum channels}
    $\Phi \in \Chan(\spc X, \spc Y)$ and $\Psi \in \Chan(\spc Z)$. The channel $\Phi$ emulates $\Psi$ if 
    \begin{equation}
        \exists \Xi_E \in \Chan(\spc Z, \spc X) \ \Xi_D \in \Chan(\spc Y, \spc Z) \ \Psi = \Xi_D \Phi \Xi_E
    \end{equation}
    The channel $\Xi_E$ is called an encoding channel and $\Xi_D$ is called a decoding channel.

    \begin{defi}
        $\Psi_0, \Psi_1 \in \Chan(\spc Z)$. $\Psi_0$ is $\epsilon$-approximation to $\Psi_1$ if
        \begin{equation}
            \vertiii{ \Psi_0 - \Psi_1 } < \epsilon
        \end{equation}
    \end{defi}

    \begin{defi}
        $\Phi \in \Chan(\spc X, \spc Y)$. $\spc Z = \C^{\{ 0,1 \}}$. $\Delta \in \Chan(\spc Z)$ is the completely dephasing channel.

        A value $\alpha > 0$ is achievable if $\forall \epsilon > 0$, for all but finitely many $n \in \N$, the channel $\Phi^{\otimes n}$ emulates an $\epsilon$-approximation to the channel $\Delta^{\floor {\alpha n}}$.

        The classical capacity of $\Phi$, denoted $\ca(\Phi)$, is the supremum value of all achievable rates for classical information transmission through $\Phi$.
    \end{defi}

    \begin{rem}
        When considering an emulation of the $m$-fold tensor product $\Delta^{\otimes m}$ of this ideal classical channel by the channel $\Phi^{\otimes n}$ , no generality is lost in restricting one's attention to classical-to-quantum encoding channels $\Xi_E$ and quantum-to-classical decoding channels $\Xi_D$.
        \begin{equation}
            \Xi_E = \Xi_E \Delta^{\otimes m} \ \ \ \ 
            \Xi_D = \Delta^{\otimes m} \Xi_D
        \end{equation}
    \end{rem}

    \begin{theo}
        $\Phi \in \Chan(\spc X, \spc Y)$
        \begin{equation}
            \ca(\Phi^{\otimes k}) = k \ \ca (\Phi)
        \end{equation}
    \end{theo}

    \begin{defi}
        Entanglement-assisted classical capacity
        \begin{equation}
            Z \mapsto (\Xi_D ( \Phi \Xi_E \otimes \I_{\Lin(\spc W)} ) ) (Z \otimes \xi)
        \end{equation}

        A value $\alpha > 0$ is achievable if $\forall \epsilon > 0$, for all but finitely many $n \in \N$, the channel $\Phi^{\otimes n}$ emulates an $\epsilon$-approximation to the channel $\Delta^{\floor {\alpha n}}$.

        The entanglement-assisted classical capacity of $\Phi$, denoted $\ca_E(\Phi)$, is the supremum value of all achievable rates for classical information transmission through $\Phi$.
    \end{defi}

    \begin{theo}
        $\Phi \in \Chan(\spc X, \spc Y)$
        \begin{equation}
            \ca_E(\Phi^{\otimes k}) = k \ca_E (\Phi)
        \end{equation}
    \end{theo}

    \begin{defi}
        $\Phi \in \Chan(\spc X, \spc Y)$. The Holevo capacity of $\Phi$ is defined as
        \begin{equation}
            \chi(\Phi) = \sup_{\eta} \chi(\Phi(\eta))
        \end{equation}
    \end{defi}

    \begin{theo}
        It is enough to maximize Holevo information using pure states. And what's more, the size of the alphabet of the ensemble can be restricted to $\dim(\spc X)^2$.
    \end{theo}
    
\end{document}
    