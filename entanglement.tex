
\documentclass[aps,pra,onecolumn,notitlepage,superscriptaddress]{revtex4-1}

%\input{myQcircuit}
\usepackage{graphicx,color}% Include figure files
\usepackage{dcolumn}% Align table columns on decimal point
\usepackage{bm}% bold math
\usepackage{amsmath,amssymb,mathrsfs}
\usepackage{url}
\usepackage{hyperref}

% added by me
\usepackage{framed}
\usepackage{algorithm}
\usepackage{dsfont}
\usepackage{mathtools}
\DeclarePairedDelimiter{\ceil}{\lceil}{\rceil}
\DeclarePairedDelimiter{\floor}{\lfloor}{\rfloor}



\newcommand{\N}{\mathbb{N}}
\newcommand{\Z}{\mathbb{Z}}
\newcommand{\R}{\mathbb{R}}
\newcommand{\C}{\mathbb{C}}


%  Sets
\newcommand{\set}[1]{\mathsf{#1}}
\newcommand{\grp}[1]{\mathsf{#1}}
\newcommand{\reg}[1]{\mathsf{#1}}
\newcommand{\spc}[1]{\mathcal{#1}}

% Integrals

\def\d{{\rm d}}

% Linear structures
\newcommand{\Span}{{\mathsf{Span}}}
\newcommand{\Lin}{\mathsf{Lin}}
\newcommand{\Pos}{\mathsf{Pos}}
\newcommand{\CP}{\mathsf{CP}}
\newcommand{\Herm}{\mathsf{Herm}}
\newcommand{\D}{\mathsf{D}}
\newcommand{\s}{\mathsf{S}}
\newcommand{\Proj}{\mathsf{Proj}}
\newcommand{\U}{\mathsf{U}}
\newcommand{\Diag}{\mathsf{Diag}}
\newcommand{\Sep}{\mathsf{Sep}}
\newcommand{\SepD}{\mathsf{SepD}}
% added by me
\newcommand{\T}{\mathsf{T}}

% added by me
\newcommand{\rank}{\mathsf{rank}}
\newcommand{\im}{\mathsf{im}}
\newcommand{\myker}{\mathsf{ker}}
% \newcommand{\Pr}{\mathsf{Pr}}

\def\>{\rangle}
\def\<{\langle}
\def\kk{\>\!\>}
\def\bb{\<\!\<}
\newcommand{\st}[1]{\mathbf{#1}}
\newcommand{\bs}[1]{\boldsymbol{#1}}

% Linear maps
\newcommand{\map}[1]{\mathcal{#1}}
\newcommand{\Tr}{\operatorname{Tr}}
\newcommand{\diag}{\mathsf{diag}}


%  Operational notions
\newcommand{\op}[1]{\operatorname{#1}}

\newcommand{\St}{{\mathsf{St}}}
\newcommand{\Eff}{{\mathsf{Eff}}}
\newcommand{\Pur}{{\mathsf{Pur}}}
\newcommand{\Transf}{{\mathsf{Transf}}}
\newcommand{\Chan}{{\mathsf{Chan}}}

%   By Mo
\newcommand{\arccot}{\mathrm{arccot}\,}

%  Miscellanea
\newcommand\myuparrow{\mathord{\uparrow}}
\newcommand\mydownarrow{\mathord{\downarrow}}
\newcommand\h{{\scriptstyle \frac 12}}
% added by me
\newcommand\I{\mathds{1}}

% Environments
\newtheorem{theo}{Theorem}
\newtheorem{ax}{Axiom}
\newtheorem{lemma}{Lemma}
\newtheorem{prop}{Proposition}
\newtheorem{cor}{Corollary}
\newtheorem{defi}{Definition}


\newtheorem{rem}{Remark}
\newtheorem{ex}{Exercise}

\newtheorem{proper}{Property}

\def\Proof{{\bf Proof.~}}
\def\qed{$\blacksquare$ \newline}

\begin{document}
    \preprint{APS/123-QED}
    \title{Bipartite entanglement}
    \author{}
    \maketitle
    % \tableofcontents
    % \newpage

    \begin{defi}
        $R \in \Sep(\spc X : \spc Y)$ if
        \begin{equation}
            R = \sum_{a \in \Sigma} P_a \otimes Q_a
        \end{equation}
        where $\{ P_a : a \in \Sigma \} \subset \Pos(\spc X)$ and $\{ Q_a : a \in \Sigma \} \subset \Pos(\spc Y)$.
    \end{defi}
    \begin{defi}
        \begin{equation}
            \SepD(\spc X : \spc Y) = \Sep(\spc X : \spc Y) \cap \D(\spc X \otimes \spc Y)
        \end{equation}
    \end{defi}
    \begin{theo}
        $\Sep(\spc X : \spc Y)$ is a convex cone and $\SepD(\spc X : \spc Y)$ is convex.
    \end{theo}
    \Proof {
        It suffices to prove that $Sep(X : Y)$ is closed under addition as well as multiplication by any nonnegative real number.
        \begin{equation}
            R_0 = \sum_{a \in \Sigma_0} P_a \otimes Q_a \ \ \ \ 
            R_1 = \sum_{a \in \Sigma_1} P_a \otimes Q_a
        \end{equation}
        Then
        \begin{equation}
            R_0 + R_1 = \sum_{a \in \Sigma_0 \cup \Sigma_1} P_a \otimes Q_a \in \Sep(\spc X : \spc Y) \ \ \ \
            \lambda R_0 = \sum_{a \in \Sigma_0} \lambda P_a \otimes Q_a \in \Sep(\spc X : \spc Y)
        \end{equation}

        $\SepD(\spc X : \spc Y)$ is convex because it is the intersection of two convex sets.
    } \qed

    \begin{lemma}
        $\spc A \subset \Pos(\spc Z)$ is a cone. $\emptyset \neq \spc B = \spc A \cap \D(\spc Z)$
        \begin{equation}
            \spc A = \op{cone}(\spc B)
        \end{equation}
    \end{lemma}
    \Proof
        \begin{enumerate}
            \item $\op{cone}(\spc B) \subset \spc A$ is obvious.
            \item Assume $P \in \spc A$, then
            \begin{equation}
                \frac{P}{\Tr(P)} \in \spc A \cap \D(\spc Z) = \spc B
            \end{equation}
        \end{enumerate}
    \qed

    \begin{theo}
        Let $\xi \in \D(\spc X \otimes \spc Y)$ be a density operator. The following statements are equivalent:
        \begin{enumerate}
            \item $\xi \in \SepD(\spc X : \spc Y)$
            \item There exists an alphabet $\Sigma$
            \begin{equation}
                \xi = \sum_{a \in \Sigma} p(a) \rho_a \otimes \sigma_a
            \end{equation}
            \item There exists an alphabet $\Sigma$
            \begin{equation}
                \xi = \sum_{a \in \Sigma} p(a) x_ax_a^* \otimes y_ay_a^*
            \end{equation}
        \end{enumerate}
    \end{theo}

    \begin{theo}
        If $\xi \in \SepD(\spc X : \spc Y)$, then there exists an alphabet $\Sigma$ with $|\Sigma| \leq \rank(\xi)^2$, $\{ x_a : a \in \Sigma \} \subset \s(\spc X)$ and $\{ y_a : a \in \Sigma \} \subset \s(\spc Y)$ such that
        \begin{equation}
            \xi = \sum_{a \in \Sigma} p(a) x_ax_a^* \otimes y_ay_a^*
        \end{equation}
    \end{theo}
    \Proof
    It holds that
    \begin{equation}
        \SepD(\spc X : \spc Y) = \op{conv} \{ xx^* \otimes yy^* : x \in \spc S(\spc X), y \in \spc S(\spc Y) \}
    \end{equation}

    And it is easy to see that
    \begin{equation}
        \xi \in \op{conv} \{ xx^* \otimes yy^* : x \in \spc S(\spc X), y \in \spc S(\spc Y), \im(xx^* \otimes yy^*) \subset \im(\xi) \}
    \end{equation}

    Notice that the following is a real affine space satisfying of dimension $\rank(\xi)^2 - 1$
    \begin{equation}
        \{ H \in \Herm(\spc X \otimes \spc Y) : \im(H) \subset \im(\xi), \Tr(H) = 1 \}
    \end{equation}

    Thus $\xi$ is contained in a affine space of dimension $\rank(\xi)^2 - 1$. This completes the proof.
    \qed

    \begin{cor}
        If $R \in \Sep(\spc X : \spc Y)$ and $R \neq 0$, then there exists an alphabet $\Sigma$ with $|\Sigma| \leq \rank(R)^2$, $\{ x_a : a \in \Sigma \} \subset \spc X$ and $\{ y_a : a \in \Sigma \} \subset \spc Y$ such that
        \begin{equation}
            R = \sum_{a \in \Sigma} x_ax_a^* \otimes y_ay_a^*
        \end{equation}
    \end{cor}

    \begin{theo}
        For every choice of complex Euclidean spaces $\spc X$ and $\spc Y$, the set $\op{SepD}(\spc X : \spc Y)$ is compact and the set $\op{Sep}(\spc X : \spc Y)$ is closed.
    \end{theo}

    \begin{theo}
        (Horodecki criterion) $R \in \Pos(\spc X \otimes \spc Y)$. The following are equivalent:
        \begin{enumerate}
            \item $R \in \Sep(\spc X : \spc Y)$
            \item If $\Phi \in \T(\spc X, \spc Z)$ is a positive map, then
            \begin{equation}
                (\Phi \otimes \I_{\Lin(\spc Y)})(R) \in \Pos(\spc Z \otimes \spc Y)
            \end{equation}
            \item If $\Phi \in \T(\spc X, \spc Y)$ is a positive unital map, then
            \begin{equation}
                (\Phi \otimes \I_{\Lin(\spc Y)})(R) \in \Pos(\spc Y \otimes \spc Y)
            \end{equation}
        \end{enumerate}
    \end{theo}
    \Proof
    It is easy to see $1 \implies 2$ and $2 \implies 3$. Thus we focus on $3 \implies 1$. 
    
    Suppose $R \notin \Sep(\spc X : \spc Y)$. As $\Sep(\spc X : \spc Y)$ is a closed, convex cone within the real vector space $\Herm(\spc X \otimes \spc Y)$, the hyperplane separation theorem implies that
    \begin{equation}
        \exists H \in \Herm(\spc X \otimes \spc Y) \ \forall S \in \Sep(\spc X : \spc Y) \ \begin{cases}
            \<H, R\> < 0 \\
            \<H, S\> \geq 0
        \end{cases}
    \end{equation}

    Such an $H$ will be used to construct a positive and unital map.

    Define $\Psi \in \T(\spc Y, \spc X)$ as
    \begin{equation}
        J(\Psi) = H
    \end{equation}

    Let $P \in \Pos(\spc X)$, $Q \in \Pos(\spc Y)$. Then
    \begin{equation}
        \<P, \Psi(Q)\> = \< P \otimes \overline Q, J(\Psi) \> = \< H, P \otimes \overline Q \> \geq 0
    \end{equation}
    Thus $\Psi$ is a positive map. It follows that $\Psi^*$ is also a positive map.

    Define $\Phi$
    \begin{equation}
        \Phi(X) = A^{-\frac 1 2} \Psi^*(X) A^{-\frac 1 2}
    \end{equation}
    where $A = \Psi^*(\I_{\spc X})$.

    Now we get a positive and unital map $\Phi$. Then we show that $(\Phi \otimes \I_{\Lin (\spc Y)})(R)$ is not positive
    \begin{align*}
        0
        &> \< H, R \> \\
        &= \< J(\Psi), R \> \\
        &= \< \op{vec}(\I_{\spc Y}) \op{vec}(\I_{\spc Y})^*, (\Psi^* \otimes \I_{\Lin (\spc Y)})(R) \> \\
        &= \< \op{vec}(\sqrt{A})\op{vec}(\sqrt{A})^*, (\Phi \otimes \I_{\Lin(\spc Y)})(R) \>
    \end{align*}

    \begin{defi}
        Define the following projectors
        \begin{align*}
            &\Delta_0 = \frac 1 n \sum_{a,b \in \Sigma} E_{a,b} \otimes E_{a,b} & \Pi_0 = \frac 1 2 \I \otimes \I + \frac 1 2 W \\
            &\Delta_1 = \I \otimes \I - \Delta_0 & \Pi_1 = \I \otimes \I - \Pi_1
        \end{align*}

        Isotropic states
        \begin{equation}
            \lambda \Delta_0 + (1-\lambda) \frac{\Delta_1}{n^2-1}
        \end{equation}

        Werner states
        \begin{equation}
            \lambda \frac{\Pi_0}{\binom{n+1}{2}} +
            (1-\lambda) \frac{\Pi_1}{\binom{n}{2}}
        \end{equation}
    \end{defi}
    
    \begin{theo}
        The isotropic state is entangled for $\lambda \in (1/n,1]$, while the Werner state is entangled for $\lambda \in [0,1/2)$.
    \end{theo}
    \Proof
    Let $\op T$ denote the transpose map
        \begin{equation}
            \op T(X) = X^T
        \end{equation}
        We have
        \begin{equation}
            (\op T \otimes \I_{\Lin(\spc Y)}) \left(
                \lambda \Delta_0 + (1-\lambda) \frac{\Delta_1}{n^2-1}
            \right) = 
                \frac{1+\lambda n}{2} \frac{\Pi_0}{\binom{n+1}{2}} +
                \frac{1-\lambda n}{2} \frac{\Pi_1}{\binom{n}{2}}
        \end{equation}
        \begin{equation}
            (\op T \otimes \I_{\Lin(\spc Y)}) \left(
                \lambda \frac{\Pi_0}{\binom{n+1}{2}} + (1-\lambda) \frac{\Pi_1}{\binom{n}{2}}
            \right) = 
                \frac{2\lambda - 1}{n} \Delta_0 +
                \left( 1- \frac{2\lambda - 1}{n} \right) \Delta_1
        \end{equation}
        
        This completes the proof. \qed

    \begin{lemma}
        \begin{equation}
            X = \sum_{a,b \in \Sigma} X_{a,b} \otimes E_{a,b} \in \Lin(\spc X \otimes \spc Y)
        \end{equation}
        We have
        \begin{equation}
            \Vert X \Vert^2 \leq \sum_{a,b \in \Sigma} \Vert X_{a,b} \Vert^2
        \end{equation}
    \end{lemma}
    \Proof
    Define
    \begin{equation}
        Y_a = \sum_{b \in \Sigma} X_{a,b} \otimes E_{a,b}
    \end{equation}
    \begin{align*}
        \Vert Y_a^*Y_a \Vert
        &= \left\Vert \sum_{b \in \Sigma} X_{a,b}X_{a,b}^* \otimes E_{a,a} \right\Vert \\
    \end{align*}
    \begin{align*}
        \Vert X \Vert^2
        &= \Vert XX^* \Vert \\
        &\leq \sum_{a \in \Sigma} \Vert Y_a^*Y_a \Vert \\
        &\leq \sum_{a,b \in \Sigma} \Vert X_{a,b}X_{a,b}^* \Vert \\
        &= \sum_{a,b \in \Sigma} \Vert X_{a,b} \Vert^2
    \end{align*}

    \begin{lemma}
        $\Phi \in \T(\spc X, \spc Y)$ is a positive and unital map. It holds that
        \begin{equation}
            \Vert \Phi(X) \Vert \leq \Vert X \Vert
        \end{equation}
    \end{lemma}

    \begin{theo}
        $H \in \Herm(\spc X \otimes \spc Y)$ and $\Vert H \Vert_2 \leq 1$, it holds that
        \begin{equation}
            \I_\spc X \otimes \I_\spc Y - H \in \Sep(\spc X : \spc Y)
        \end{equation}
    \end{theo}
\end{document}
    